\documentclass[9pt,]{article}
\usepackage{lmodern}
\usepackage{amssymb,amsmath}
\usepackage{ifxetex,ifluatex}
\usepackage{fixltx2e} % provides \textsubscript
\ifnum 0\ifxetex 1\fi\ifluatex 1\fi=0 % if pdftex
  \usepackage[T1]{fontenc}
  \usepackage[utf8]{inputenc}
\else % if luatex or xelatex
  \ifxetex
    \usepackage{mathspec}
  \else
    \usepackage{fontspec}
  \fi
  \defaultfontfeatures{Ligatures=TeX,Scale=MatchLowercase}
\fi
% use upquote if available, for straight quotes in verbatim environments
\IfFileExists{upquote.sty}{\usepackage{upquote}}{}
% use microtype if available
\IfFileExists{microtype.sty}{%
\usepackage{microtype}
\UseMicrotypeSet[protrusion]{basicmath} % disable protrusion for tt fonts
}{}
\usepackage[margin=1in]{geometry}
\usepackage{hyperref}
\hypersetup{unicode=true,
            pdftitle={Test 2},
            pdfborder={0 0 0},
            breaklinks=true}
\urlstyle{same}  % don't use monospace font for urls
\usepackage{color}
\usepackage{fancyvrb}
\newcommand{\VerbBar}{|}
\newcommand{\VERB}{\Verb[commandchars=\\\{\}]}
\DefineVerbatimEnvironment{Highlighting}{Verbatim}{commandchars=\\\{\}}
% Add ',fontsize=\small' for more characters per line
\usepackage{framed}
\definecolor{shadecolor}{RGB}{248,248,248}
\newenvironment{Shaded}{\begin{snugshade}}{\end{snugshade}}
\newcommand{\KeywordTok}[1]{\textcolor[rgb]{0.13,0.29,0.53}{\textbf{#1}}}
\newcommand{\DataTypeTok}[1]{\textcolor[rgb]{0.13,0.29,0.53}{#1}}
\newcommand{\DecValTok}[1]{\textcolor[rgb]{0.00,0.00,0.81}{#1}}
\newcommand{\BaseNTok}[1]{\textcolor[rgb]{0.00,0.00,0.81}{#1}}
\newcommand{\FloatTok}[1]{\textcolor[rgb]{0.00,0.00,0.81}{#1}}
\newcommand{\ConstantTok}[1]{\textcolor[rgb]{0.00,0.00,0.00}{#1}}
\newcommand{\CharTok}[1]{\textcolor[rgb]{0.31,0.60,0.02}{#1}}
\newcommand{\SpecialCharTok}[1]{\textcolor[rgb]{0.00,0.00,0.00}{#1}}
\newcommand{\StringTok}[1]{\textcolor[rgb]{0.31,0.60,0.02}{#1}}
\newcommand{\VerbatimStringTok}[1]{\textcolor[rgb]{0.31,0.60,0.02}{#1}}
\newcommand{\SpecialStringTok}[1]{\textcolor[rgb]{0.31,0.60,0.02}{#1}}
\newcommand{\ImportTok}[1]{#1}
\newcommand{\CommentTok}[1]{\textcolor[rgb]{0.56,0.35,0.01}{\textit{#1}}}
\newcommand{\DocumentationTok}[1]{\textcolor[rgb]{0.56,0.35,0.01}{\textbf{\textit{#1}}}}
\newcommand{\AnnotationTok}[1]{\textcolor[rgb]{0.56,0.35,0.01}{\textbf{\textit{#1}}}}
\newcommand{\CommentVarTok}[1]{\textcolor[rgb]{0.56,0.35,0.01}{\textbf{\textit{#1}}}}
\newcommand{\OtherTok}[1]{\textcolor[rgb]{0.56,0.35,0.01}{#1}}
\newcommand{\FunctionTok}[1]{\textcolor[rgb]{0.00,0.00,0.00}{#1}}
\newcommand{\VariableTok}[1]{\textcolor[rgb]{0.00,0.00,0.00}{#1}}
\newcommand{\ControlFlowTok}[1]{\textcolor[rgb]{0.13,0.29,0.53}{\textbf{#1}}}
\newcommand{\OperatorTok}[1]{\textcolor[rgb]{0.81,0.36,0.00}{\textbf{#1}}}
\newcommand{\BuiltInTok}[1]{#1}
\newcommand{\ExtensionTok}[1]{#1}
\newcommand{\PreprocessorTok}[1]{\textcolor[rgb]{0.56,0.35,0.01}{\textit{#1}}}
\newcommand{\AttributeTok}[1]{\textcolor[rgb]{0.77,0.63,0.00}{#1}}
\newcommand{\RegionMarkerTok}[1]{#1}
\newcommand{\InformationTok}[1]{\textcolor[rgb]{0.56,0.35,0.01}{\textbf{\textit{#1}}}}
\newcommand{\WarningTok}[1]{\textcolor[rgb]{0.56,0.35,0.01}{\textbf{\textit{#1}}}}
\newcommand{\AlertTok}[1]{\textcolor[rgb]{0.94,0.16,0.16}{#1}}
\newcommand{\ErrorTok}[1]{\textcolor[rgb]{0.64,0.00,0.00}{\textbf{#1}}}
\newcommand{\NormalTok}[1]{#1}
\usepackage{graphicx,grffile}
\makeatletter
\def\maxwidth{\ifdim\Gin@nat@width>\linewidth\linewidth\else\Gin@nat@width\fi}
\def\maxheight{\ifdim\Gin@nat@height>\textheight\textheight\else\Gin@nat@height\fi}
\makeatother
% Scale images if necessary, so that they will not overflow the page
% margins by default, and it is still possible to overwrite the defaults
% using explicit options in \includegraphics[width, height, ...]{}
\setkeys{Gin}{width=\maxwidth,height=\maxheight,keepaspectratio}
\IfFileExists{parskip.sty}{%
\usepackage{parskip}
}{% else
\setlength{\parindent}{0pt}
\setlength{\parskip}{6pt plus 2pt minus 1pt}
}
\setlength{\emergencystretch}{3em}  % prevent overfull lines
\providecommand{\tightlist}{%
  \setlength{\itemsep}{0pt}\setlength{\parskip}{0pt}}
\setcounter{secnumdepth}{0}
% Redefines (sub)paragraphs to behave more like sections
\ifx\paragraph\undefined\else
\let\oldparagraph\paragraph
\renewcommand{\paragraph}[1]{\oldparagraph{#1}\mbox{}}
\fi
\ifx\subparagraph\undefined\else
\let\oldsubparagraph\subparagraph
\renewcommand{\subparagraph}[1]{\oldsubparagraph{#1}\mbox{}}
\fi

%%% Use protect on footnotes to avoid problems with footnotes in titles
\let\rmarkdownfootnote\footnote%
\def\footnote{\protect\rmarkdownfootnote}

%%% Change title format to be more compact
\usepackage{titling}

% Create subtitle command for use in maketitle
\newcommand{\subtitle}[1]{
  \posttitle{
    \begin{center}\large#1\end{center}
    }
}

\setlength{\droptitle}{-2em}

  \title{Test 2}
    \pretitle{\vspace{\droptitle}\centering\huge}
  \posttitle{\par}
    \author{}
    \preauthor{}\postauthor{}
    \date{}
    \predate{}\postdate{}
  

\begin{document}
\maketitle

Loading the packages first

\begin{Shaded}
\begin{Highlighting}[]
\KeywordTok{library}\NormalTok{(pacman)}
    \KeywordTok{p_load}\NormalTok{(tidyverse,dplyr, lavaan, psych, knitr, lme4, lmerTest, multilevel, nlme, lattice, sjPlot, ggplot2,cowplot, magrittr, broom, Metrics, pbkrtest, mlmRev, influence.ME, gridExtra, semPlot, plyr)}
\end{Highlighting}
\end{Shaded}

Question 1 -- Structural Equation Modeling

loading data

\begin{Shaded}
\begin{Highlighting}[]
\NormalTok{data <-}\StringTok{ }\KeywordTok{readRDS}\NormalTok{(}\StringTok{"peru2.RDS"}\NormalTok{)  }
\end{Highlighting}
\end{Shaded}

Making variables for the path analysis for SEM. I would like to predict
risky/antisocial behaviour using self-efficacy, early life SES (round 1
and 2), mental health and agency

\begin{Shaded}
\begin{Highlighting}[]
\CommentTok{#   SES (Using the following variables: Housing quality, Access to services, Consumer durables)}
\NormalTok{    data}\OperatorTok{$}\NormalTok{hq_r12<-data}\OperatorTok{$}\NormalTok{hq_r12}\OperatorTok{/}\DecValTok{3}
\NormalTok{    data}\OperatorTok{$}\NormalTok{sv_r12<-data}\OperatorTok{$}\NormalTok{sv_r12}\OperatorTok{/}\DecValTok{3}
\NormalTok{    data}\OperatorTok{$}\NormalTok{cd_r12<-data}\OperatorTok{$}\NormalTok{cd_r12}\OperatorTok{/}\DecValTok{3}
\NormalTok{    data}\OperatorTok{$}\NormalTok{ses_scale<-data}\OperatorTok{$}\NormalTok{hq_r12}\OperatorTok{+}\NormalTok{data}\OperatorTok{$}\NormalTok{sv_r12}\OperatorTok{+}\NormalTok{data}\OperatorTok{$}\NormalTok{cd_r12}
    
\CommentTok{#   self-efficacy}
\NormalTok{    data}\OperatorTok{$}\NormalTok{self_eff1<-(data}\OperatorTok{$}\NormalTok{self_eff1)}\OperatorTok{/}\DecValTok{8}    
\NormalTok{    data}\OperatorTok{$}\NormalTok{self_eff2<-(data}\OperatorTok{$}\NormalTok{self_eff2)}\OperatorTok{/}\DecValTok{8}  
\NormalTok{    data}\OperatorTok{$}\NormalTok{self_eff3<-(data}\OperatorTok{$}\NormalTok{self_eff3)}\OperatorTok{/}\DecValTok{8}
\NormalTok{    data}\OperatorTok{$}\NormalTok{self_eff4<-(data}\OperatorTok{$}\NormalTok{self_eff4)}\OperatorTok{/}\DecValTok{8}
\NormalTok{    data}\OperatorTok{$}\NormalTok{self_eff5<-(data}\OperatorTok{$}\NormalTok{self_eff5)}\OperatorTok{/}\DecValTok{8}
\NormalTok{    data}\OperatorTok{$}\NormalTok{self_eff6<-(data}\OperatorTok{$}\NormalTok{self_eff6)}\OperatorTok{/}\DecValTok{8}
\NormalTok{    data}\OperatorTok{$}\NormalTok{self_eff7<-(data}\OperatorTok{$}\NormalTok{self_eff7)}\OperatorTok{/}\DecValTok{8}
\NormalTok{    data}\OperatorTok{$}\NormalTok{self_eff8<-(data}\OperatorTok{$}\NormalTok{self_eff8)}\OperatorTok{/}\DecValTok{8}
\NormalTok{    data}\OperatorTok{$}\NormalTok{self_efficacy_scale<-(data}\OperatorTok{$}\NormalTok{self_eff1 }\OperatorTok{+}\StringTok{ }\NormalTok{data}\OperatorTok{$}\NormalTok{self_eff2 }\OperatorTok{+}\StringTok{ }\NormalTok{data}\OperatorTok{$}\NormalTok{self_eff3 }\OperatorTok{+}\StringTok{ }\NormalTok{data}\OperatorTok{$}\NormalTok{self_eff4 }\OperatorTok{+}\StringTok{ }\NormalTok{data}\OperatorTok{$}\NormalTok{self_eff5 }\OperatorTok{+}\StringTok{ }\NormalTok{data}\OperatorTok{$}\NormalTok{self_eff6 }\OperatorTok{+}\StringTok{ }\NormalTok{data}\OperatorTok{$}\NormalTok{self_eff7 }\OperatorTok{+}\StringTok{ }\NormalTok{data}\OperatorTok{$}\NormalTok{self_eff8)}
    
\CommentTok{#   agency}
\NormalTok{    data}\OperatorTok{$}\NormalTok{agency1<-data}\OperatorTok{$}\NormalTok{agency1}\OperatorTok{/}\DecValTok{4}
\NormalTok{    data}\OperatorTok{$}\NormalTok{agency2<-data}\OperatorTok{$}\NormalTok{agency2}\OperatorTok{/}\DecValTok{4}
\NormalTok{    data}\OperatorTok{$}\NormalTok{agency3<-data}\OperatorTok{$}\NormalTok{agency3}\OperatorTok{/}\DecValTok{4}
\NormalTok{    data}\OperatorTok{$}\NormalTok{agency4<-data}\OperatorTok{$}\NormalTok{agency4}\OperatorTok{/}\DecValTok{4}
\NormalTok{    data}\OperatorTok{$}\NormalTok{agency_scale<-data}\OperatorTok{$}\NormalTok{agency1}\OperatorTok{+}\NormalTok{data}\OperatorTok{$}\NormalTok{agency2}\OperatorTok{+}\NormalTok{data}\OperatorTok{$}\NormalTok{agency3}\OperatorTok{+}\NormalTok{data}\OperatorTok{$}\NormalTok{agency4}
    
\CommentTok{#   mental wellness}
\NormalTok{    data}\OperatorTok{$}\NormalTok{sdq1<-data}\OperatorTok{$}\NormalTok{sdq1}\OperatorTok{/}\DecValTok{5}
\NormalTok{    data}\OperatorTok{$}\NormalTok{sdq2<-data}\OperatorTok{$}\NormalTok{sdq2}\OperatorTok{/}\DecValTok{5}
\NormalTok{    data}\OperatorTok{$}\NormalTok{sdq3<-data}\OperatorTok{$}\NormalTok{sdq3}\OperatorTok{/}\DecValTok{5}
\NormalTok{    data}\OperatorTok{$}\NormalTok{sdq4<-data}\OperatorTok{$}\NormalTok{sdq4}\OperatorTok{/}\DecValTok{5}
\NormalTok{    data}\OperatorTok{$}\NormalTok{sdq5<-data}\OperatorTok{$}\NormalTok{sdq5}\OperatorTok{/}\DecValTok{5}
\NormalTok{    data}\OperatorTok{$}\NormalTok{mental_wellness_scale<-data}\OperatorTok{$}\NormalTok{sdq1}\OperatorTok{+}\NormalTok{data}\OperatorTok{$}\NormalTok{sdq2}\OperatorTok{+}\NormalTok{data}\OperatorTok{$}\NormalTok{sdq3}\OperatorTok{+}\NormalTok{data}\OperatorTok{$}\NormalTok{sdq4}\OperatorTok{+}\NormalTok{data}\OperatorTok{$}\NormalTok{sdq5}
    
\CommentTok{#   anti-social, or risk-taking behaviour at Round 5 }
    
\NormalTok{    data}\OperatorTok{$}\NormalTok{FRNSMKR5<-data}\OperatorTok{$}\NormalTok{FRNSMKR5}\OperatorTok{/}\DecValTok{9} \CommentTok{#Have friends who smoke   }
\NormalTok{    data}\OperatorTok{$}\NormalTok{FRNALCR5<-data}\OperatorTok{$}\NormalTok{FRNALCR5}\OperatorTok{/}\DecValTok{9} \CommentTok{#Have friends who use alcohol}
\NormalTok{    data}\OperatorTok{$}\NormalTok{YOUALCR5<-data}\OperatorTok{$}\NormalTok{YOUALCR5}\OperatorTok{/}\DecValTok{9} \CommentTok{#Uses alcohol }
\NormalTok{    data}\OperatorTok{$}\NormalTok{BEATEN<-data}\OperatorTok{$}\NormalTok{BEATEN}\OperatorTok{/}\DecValTok{9}     \CommentTok{#Beaten up by friends, strangers, teachers, parents }
\NormalTok{    data}\OperatorTok{$}\NormalTok{ARRSTDR5<-data}\OperatorTok{$}\NormalTok{ARRSTDR5}\OperatorTok{/}\DecValTok{9} \CommentTok{#Has been arrested}
\NormalTok{    data}\OperatorTok{$}\NormalTok{FRNGNGR5<-data}\OperatorTok{$}\NormalTok{FRNGNGR5}\OperatorTok{/}\DecValTok{9} \CommentTok{#Has friends in gang}
\NormalTok{    data}\OperatorTok{$}\NormalTok{MEMGNGR5<-data}\OperatorTok{$}\NormalTok{MEMGNGR5}\OperatorTok{/}\DecValTok{9} \CommentTok{#Is a member of a gang}
\NormalTok{    data}\OperatorTok{$}\NormalTok{CRYWPNR5<-data}\OperatorTok{$}\NormalTok{CRYWPNR5}\OperatorTok{/}\DecValTok{9} \CommentTok{#Has carried a weapon}
\NormalTok{    data}\OperatorTok{$}\NormalTok{NUMPRTR5<-data}\OperatorTok{$}\NormalTok{NUMPRTR5}\OperatorTok{/}\DecValTok{9} \CommentTok{#Number of sex partners had}
\NormalTok{    data}\OperatorTok{$}\NormalTok{risky_behaviour_scale<-(data}\OperatorTok{$}\NormalTok{FRNSMKR5 }\OperatorTok{+}\StringTok{ }\NormalTok{data}\OperatorTok{$}\NormalTok{FRNALCR5 }\OperatorTok{+}\StringTok{ }\NormalTok{data}\OperatorTok{$}\NormalTok{YOUALCR5 }\OperatorTok{+}\StringTok{ }\NormalTok{data}\OperatorTok{$}\NormalTok{BEATEN }\OperatorTok{+}\StringTok{ }\NormalTok{data}\OperatorTok{$}\NormalTok{ARRSTDR5 }\OperatorTok{+}\StringTok{ }\NormalTok{data}\OperatorTok{$}\NormalTok{FRNGNGR5 }\OperatorTok{+}\StringTok{ }\NormalTok{data}\OperatorTok{$}\NormalTok{MEMGNGR5 }\OperatorTok{+}\StringTok{ }\NormalTok{data}\OperatorTok{$}\NormalTok{CRYWPNR5 }\OperatorTok{+}\StringTok{ }\NormalTok{data}\OperatorTok{$}\NormalTok{NUMPRTR5)}
\end{Highlighting}
\end{Shaded}

Defining path model

\begin{Shaded}
\begin{Highlighting}[]
\NormalTok{pathm1 <-}\StringTok{ '}
\StringTok{      risky_behaviour_scale ~ self_efficacy_scale}
\StringTok{      mental_wellness_scale ~ self_efficacy_scale}

\StringTok{'}
\CommentTok{#Fitting and summarising the model}
\NormalTok{prejpathfit1 <-}\StringTok{ }\KeywordTok{sem}\NormalTok{(pathm1, }\DataTypeTok{data =}\NormalTok{ data)}
\KeywordTok{summary}\NormalTok{(prejpathfit1, }\DataTypeTok{fit.measures =}\NormalTok{ T) }
\end{Highlighting}
\end{Shaded}

\begin{verbatim}
## lavaan 0.6-3 ended normally after 14 iterations
## 
##   Optimization method                           NLMINB
##   Number of free parameters                          5
## 
##                                                   Used       Total
##   Number of observations                          1559        1860
## 
##   Estimator                                         ML
##   Model Fit Test Statistic                       0.000
##   Degrees of freedom                                 0
##   Minimum Function Value               0.0000000000000
## 
## Model test baseline model:
## 
##   Minimum Function Test Statistic              109.597
##   Degrees of freedom                                 3
##   P-value                                        0.000
## 
## User model versus baseline model:
## 
##   Comparative Fit Index (CFI)                    1.000
##   Tucker-Lewis Index (TLI)                       1.000
## 
## Loglikelihood and Information Criteria:
## 
##   Loglikelihood user model (H0)              -1559.863
##   Loglikelihood unrestricted model (H1)      -1559.863
## 
##   Number of free parameters                          5
##   Akaike (AIC)                                3129.727
##   Bayesian (BIC)                              3156.486
##   Sample-size adjusted Bayesian (BIC)         3140.602
## 
## Root Mean Square Error of Approximation:
## 
##   RMSEA                                          0.000
##   90 Percent Confidence Interval          0.000  0.000
##   P-value RMSEA <= 0.05                             NA
## 
## Standardized Root Mean Square Residual:
## 
##   SRMR                                           0.000
## 
## Parameter Estimates:
## 
##   Information                                 Expected
##   Information saturated (h1) model          Structured
##   Standard Errors                             Standard
## 
## Regressions:
##                           Estimate  Std.Err  z-value  P(>|z|)
##   risky_behaviour_scale ~                                    
##     self_ffccy_scl          -0.044    0.027   -1.642    0.101
##   mental_wellness_scale ~                                    
##     self_ffccy_scl           0.172    0.042    4.107    0.000
## 
## Covariances:
##                            Estimate  Std.Err  z-value  P(>|z|)
##  .risky_behaviour_scale ~~                                    
##    .mntl_wllnss_sc           -0.039    0.004   -9.106    0.000
## 
## Variances:
##                    Estimate  Std.Err  z-value  P(>|z|)
##    .risky_bhvr_scl    0.106    0.004   27.920    0.000
##    .mntl_wllnss_sc    0.253    0.009   27.920    0.000
\end{verbatim}

CFI and TLI are \textgreater{} .09 which shows that the model has a good
fit, The model is shows a good fit, not too sure about the RMSEA,
p-value =Na, might need to compare to another model

Now drawing model using package semPlot

\begin{Shaded}
\begin{Highlighting}[]
\NormalTok{semPlot}\OperatorTok{::}\KeywordTok{semPaths}\NormalTok{(prejpathfit1, }\DataTypeTok{what =} \StringTok{"est"}\NormalTok{, }\DataTypeTok{layout =} \StringTok{"spring"}\NormalTok{)}
\end{Highlighting}
\end{Shaded}

\includegraphics{Test_2_files/figure-latex/unnamed-chunk-3-1.pdf} Our
model is telling us that self efficacy and mental wellness are
predictors of risky behaviours with self-efficacy also predicting mental
illness

Defining path model 2

\begin{Shaded}
\begin{Highlighting}[]
\NormalTok{pathm2 <-}\StringTok{ '}
\StringTok{      risky_behaviour_scale ~ self_efficacy_scale + mental_wellness_scale}
\StringTok{      mental_wellness_scale ~ self_efficacy_scale + agency_scale}
\StringTok{      self_efficacy_scale ~  agency_scale }

\StringTok{'}  

\CommentTok{#Fitting and summarising model}
\NormalTok{prejpathfit2 <-}\StringTok{ }\KeywordTok{sem}\NormalTok{(pathm2, }\DataTypeTok{data =}\NormalTok{ data)}
\KeywordTok{summary}\NormalTok{(prejpathfit2, }\DataTypeTok{fit.measures =}\NormalTok{ T) }
\end{Highlighting}
\end{Shaded}

\begin{verbatim}
## lavaan 0.6-3 ended normally after 20 iterations
## 
##   Optimization method                           NLMINB
##   Number of free parameters                          8
## 
##                                                   Used       Total
##   Number of observations                          1523        1860
## 
##   Estimator                                         ML
##   Model Fit Test Statistic                       2.076
##   Degrees of freedom                                 1
##   P-value (Chi-square)                           0.150
## 
## Model test baseline model:
## 
##   Minimum Function Test Statistic              445.819
##   Degrees of freedom                                 6
##   P-value                                        0.000
## 
## User model versus baseline model:
## 
##   Comparative Fit Index (CFI)                    0.998
##   Tucker-Lewis Index (TLI)                       0.985
## 
## Loglikelihood and Information Criteria:
## 
##   Loglikelihood user model (H0)              -1689.030
##   Loglikelihood unrestricted model (H1)      -1687.992
## 
##   Number of free parameters                          8
##   Akaike (AIC)                                3394.060
##   Bayesian (BIC)                              3436.687
##   Sample-size adjusted Bayesian (BIC)         3411.273
## 
## Root Mean Square Error of Approximation:
## 
##   RMSEA                                          0.027
##   90 Percent Confidence Interval          0.000  0.079
##   P-value RMSEA <= 0.05                          0.695
## 
## Standardized Root Mean Square Residual:
## 
##   SRMR                                           0.010
## 
## Parameter Estimates:
## 
##   Information                                 Expected
##   Information saturated (h1) model          Structured
##   Standard Errors                             Standard
## 
## Regressions:
##                           Estimate  Std.Err  z-value  P(>|z|)
##   risky_behaviour_scale ~                                    
##     self_ffccy_scl          -0.010    0.026   -0.370    0.712
##     mntl_wllnss_sc          -0.149    0.016   -9.312    0.000
##   mental_wellness_scale ~                                    
##     self_ffccy_scl           0.259    0.047    5.552    0.000
##     agency_scale            -0.168    0.034   -4.913    0.000
##   self_efficacy_scale ~                                      
##     agency_scale             0.320    0.017   18.866    0.000
## 
## Variances:
##                    Estimate  Std.Err  z-value  P(>|z|)
##    .risky_bhvr_scl    0.098    0.004   27.595    0.000
##    .mntl_wllnss_sc    0.249    0.009   27.595    0.000
##    .self_ffccy_scl    0.075    0.003   27.595    0.000
\end{verbatim}

\begin{Shaded}
\begin{Highlighting}[]
\CommentTok{#comparing this model to our first model}
\KeywordTok{anova}\NormalTok{(prejpathfit2, prejpathfit1) }
\end{Highlighting}
\end{Shaded}

\begin{verbatim}
## Chi Square Difference Test
## 
##              Df    AIC    BIC  Chisq Chisq diff Df diff Pr(>Chisq)
## prejpathfit1  0 3129.7 3156.5 0.0000                              
## prejpathfit2  1 3394.1 3436.7 2.0764     2.0764       1     0.1496
\end{verbatim}

Now plotting this model

\begin{Shaded}
\begin{Highlighting}[]
\NormalTok{semPlot}\OperatorTok{::}\KeywordTok{semPaths}\NormalTok{(prejpathfit2, }\DataTypeTok{what =} \StringTok{"est"}\NormalTok{, }\DataTypeTok{layout =} \StringTok{"spring"}\NormalTok{)}
\end{Highlighting}
\end{Shaded}

\includegraphics{Test_2_files/figure-latex/unnamed-chunk-4-1.pdf} Let's
take a closer look at the residuals as our new model as the AIC is
higher than the first model's one and the RMSEA p-value is not
significant

\begin{Shaded}
\begin{Highlighting}[]
      \KeywordTok{resid}\NormalTok{(prejpathfit2, }\DataTypeTok{type =} \StringTok{"normalized"}\NormalTok{)}
\end{Highlighting}
\end{Shaded}

\begin{verbatim}
## $type
## [1] "normalized"
## 
## $cov
##                       rsky__ mntl__ slf_f_ agncy_
## risky_behaviour_scale  0.000                     
## mental_wellness_scale  0.000  0.000              
## self_efficacy_scale    0.000  0.000  0.000       
## agency_scale          -1.251  0.000  0.000  0.000
\end{verbatim}

\begin{Shaded}
\begin{Highlighting}[]
      \KeywordTok{modificationindices}\NormalTok{(prejpathfit2)}
\end{Highlighting}
\end{Shaded}

\begin{verbatim}
##                      lhs op                   rhs    mi    epc sepc.lv
## 10 risky_behaviour_scale ~~ mental_wellness_scale 2.075 -0.046  -0.046
## 11 risky_behaviour_scale ~~   self_efficacy_scale 2.075  0.007   0.007
## 13 risky_behaviour_scale  ~          agency_scale 2.075 -0.031  -0.031
## 14 mental_wellness_scale  ~ risky_behaviour_scale 2.075 -0.470  -0.470
## 15   self_efficacy_scale  ~ risky_behaviour_scale 2.075  0.075   0.075
## 17          agency_scale  ~ risky_behaviour_scale 2.075 -0.055  -0.055
##    sepc.all sepc.nox
## 10   -0.296   -0.296
## 11    0.085    0.085
## 13   -0.040   -0.097
## 14   -0.300   -0.300
## 15    0.079    0.079
## 17   -0.043   -0.043
\end{verbatim}

we can see correlation between the risky behaviour scale and the mental
wellness and self-efficacy scales now lets respecify the model to allow
correlations between the error terms in each scale

\begin{Shaded}
\begin{Highlighting}[]
\NormalTok{  prejpathmodel2 <-}\StringTok{ '}
\StringTok{                # MEASUREMENT MODEL}
\StringTok{                  agency_scale =~ agency1 + agency2 + agency3 + agency4}
\StringTok{                  self_efficacy_scale =~ self_eff1 + self_eff2 + self_eff3 + self_eff4 + self_eff5 + self_eff6                   + self_eff7 + self_eff8}
\StringTok{                  mental_wellness_scale =~ sdq1 + sdq2 + sdq3 + sdq4 + sdq5}
\StringTok{                  risky_behaviour_scale =~ FRNSMKR5 + FRNALCR5 + YOUALCR5 + BEATEN + ARRSTDR5 + FRNGNGR5 +                      MEMGNGR5 + CRYWPNR5 + NUMPRTR5}

\StringTok{      }
\StringTok{                # STRUCTURAL MODEL}
\StringTok{                  risky_behaviour_scale ~ self_efficacy_scale + mental_wellness_scale}
\StringTok{                  mental_wellness_scale ~ self_efficacy_scale + agency_scale}
\StringTok{                  self_efficacy_scale ~  agency_scale }
\StringTok{                  }
\StringTok{                # CORRELATED ERRORS}
\StringTok{                  risky_behaviour_scale ~~  mental_wellness_scale}
\StringTok{                  risky_behaviour_scale ~~  self_efficacy_scale}
\StringTok{      }
\StringTok{               '}
\CommentTok{#Fitting and summarising model}
\NormalTok{ prejpathfit3 <-}\StringTok{ }\KeywordTok{sem}\NormalTok{(prejpathmodel2, }\DataTypeTok{data =}\NormalTok{ data)}
      \KeywordTok{summary}\NormalTok{(prejpathfit3, }\DataTypeTok{fit.measures =}\NormalTok{ T)}
\end{Highlighting}
\end{Shaded}

\begin{verbatim}
## lavaan 0.6-3 ended normally after 328 iterations
## 
##   Optimization method                           NLMINB
##   Number of free parameters                         59
## 
##                                                   Used       Total
##   Number of observations                          1523        1860
## 
##   Estimator                                         ML
##   Model Fit Test Statistic                    1114.457
##   Degrees of freedom                               292
##   P-value (Chi-square)                           0.000
## 
## Model test baseline model:
## 
##   Minimum Function Test Statistic             6745.620
##   Degrees of freedom                               325
##   P-value                                        0.000
## 
## User model versus baseline model:
## 
##   Comparative Fit Index (CFI)                    0.872
##   Tucker-Lewis Index (TLI)                       0.857
## 
## Loglikelihood and Information Criteria:
## 
##   Loglikelihood user model (H0)              46244.875
##   Loglikelihood unrestricted model (H1)      46802.104
## 
##   Number of free parameters                         59
##   Akaike (AIC)                              -92371.750
##   Bayesian (BIC)                            -92057.372
##   Sample-size adjusted Bayesian (BIC)       -92244.800
## 
## Root Mean Square Error of Approximation:
## 
##   RMSEA                                          0.043
##   90 Percent Confidence Interval          0.040  0.046
##   P-value RMSEA <= 0.05                          1.000
## 
## Standardized Root Mean Square Residual:
## 
##   SRMR                                           0.043
## 
## Parameter Estimates:
## 
##   Information                                 Expected
##   Information saturated (h1) model          Structured
##   Standard Errors                             Standard
## 
## Latent Variables:
##                            Estimate  Std.Err  z-value  P(>|z|)
##   agency_scale =~                                             
##     agency1                   1.000                           
##     agency2                   0.086       NA                  
##     agency3                   1.124       NA                  
##     agency4                   1.065       NA                  
##   self_efficacy_scale =~                                      
##     self_eff1                 1.000                           
##     self_eff2                 1.210       NA                  
##     self_eff3                 1.115       NA                  
##     self_eff4                 0.819       NA                  
##     self_eff5                 1.023       NA                  
##     self_eff6                 0.847       NA                  
##     self_eff7                 0.937       NA                  
##     self_eff8                 1.226       NA                  
##   mental_wellness_scale =~                                    
##     sdq1                      1.000                           
##     sdq2                      1.042       NA                  
##     sdq3                      1.270       NA                  
##     sdq4                      1.063       NA                  
##     sdq5                      1.241       NA                  
##   risky_behaviour_scale =~                                    
##     FRNSMKR5                  1.000                           
##     FRNALCR5                  1.325       NA                  
##     YOUALCR5                  1.214       NA                  
##     BEATEN                    0.622       NA                  
##     ARRSTDR5                  0.104       NA                  
##     FRNGNGR5                  0.709       NA                  
##     MEMGNGR5                  0.158       NA                  
##     CRYWPNR5                  0.187       NA                  
##     NUMPRTR5                  0.742       NA                  
## 
## Regressions:
##                           Estimate  Std.Err  z-value  P(>|z|)
##   risky_behaviour_scale ~                                    
##     self_ffccy_scl           0.027       NA                  
##     mntl_wllnss_sc          -0.005       NA                  
##   mental_wellness_scale ~                                    
##     self_ffccy_scl           1.204       NA                  
##     agency_scale            -0.449       NA                  
##   self_efficacy_scale ~                                      
##     agency_scale             0.308       NA                  
## 
## Covariances:
##                            Estimate  Std.Err  z-value  P(>|z|)
##  .mental_wellness_scale ~~                                    
##    .risky_bhvr_scl           -0.001       NA                  
##  .self_efficacy_scale ~~                                      
##    .risky_bhvr_scl           -0.000       NA                  
## 
## Variances:
##                    Estimate  Std.Err  z-value  P(>|z|)
##    .agency1           0.015       NA                  
##    .agency2           0.053       NA                  
##    .agency3           0.020       NA                  
##    .agency4           0.015       NA                  
##    .self_eff1         0.003       NA                  
##    .self_eff2         0.004       NA                  
##    .self_eff3         0.003       NA                  
##    .self_eff4         0.003       NA                  
##    .self_eff5         0.002       NA                  
##    .self_eff6         0.003       NA                  
##    .self_eff7         0.003       NA                  
##    .self_eff8         0.003       NA                  
##    .sdq1              0.012       NA                  
##    .sdq2              0.013       NA                  
##    .sdq3              0.012       NA                  
##    .sdq4              0.016       NA                  
##    .sdq5              0.013       NA                  
##    .FRNSMKR5          0.004       NA                  
##    .FRNALCR5          0.004       NA                  
##    .YOUALCR5          0.008       NA                  
##    .BEATEN            0.006       NA                  
##    .ARRSTDR5          0.000       NA                  
##    .FRNGNGR5          0.003       NA                  
##    .MEMGNGR5          0.000       NA                  
##    .CRYWPNR5          0.001       NA                  
##    .NUMPRTR5          0.005       NA                  
##     agency_scale      0.007       NA                  
##    .self_ffccy_scl    0.000       NA                  
##    .mntl_wllnss_sc    0.005       NA                  
##    .risky_bhvr_scl    0.002       NA
\end{verbatim}

Although the RMSEA, SRMR and AIC/BIC are suggestive of a better fitting
model, the RMSEA p-value is 1???

Now plotting this model

\begin{Shaded}
\begin{Highlighting}[]
\NormalTok{semPlot}\OperatorTok{::}\KeywordTok{semPaths}\NormalTok{(prejpathfit3, }\DataTypeTok{what =} \StringTok{"est"}\NormalTok{, }\DataTypeTok{layout =} \StringTok{"spring"}\NormalTok{)}
\end{Highlighting}
\end{Shaded}

\includegraphics{Test_2_files/figure-latex/unnamed-chunk-7-1.pdf} With
the correlated errors we can see that the riscky behaviour scale is not
linked to any of the other variables Let's see waht happens if the
correlated errors are excluded

\begin{Shaded}
\begin{Highlighting}[]
\NormalTok{prejpathmodel3 <-}\StringTok{ '}
\StringTok{                # MEASUREMENT MODEL}
\StringTok{                  agency_scale =~ agency1 + agency2 + agency3 + agency4}
\StringTok{                  self_efficacy_scale =~ self_eff1 + self_eff2 + self_eff3 + self_eff4 + self_eff5 + self_eff6                   + self_eff7 + self_eff8}
\StringTok{                  mental_wellness_scale =~ sdq1 + sdq2 + sdq3 + sdq4 + sdq5}
\StringTok{                  risky_behaviour_scale =~ FRNSMKR5 + FRNALCR5 + YOUALCR5 + BEATEN + ARRSTDR5 + FRNGNGR5 +                      MEMGNGR5 + CRYWPNR5 + NUMPRTR5}

\StringTok{      }
\StringTok{                # STRUCTURAL MODEL}
\StringTok{                  risky_behaviour_scale ~ self_efficacy_scale + mental_wellness_scale}
\StringTok{                  mental_wellness_scale ~ self_efficacy_scale + agency_scale}
\StringTok{                  self_efficacy_scale ~  agency_scale }
\StringTok{                  }
\StringTok{               }
\StringTok{               '}
\CommentTok{#Fitting and summarising model}
\NormalTok{ prejpathfit4 <-}\StringTok{ }\KeywordTok{sem}\NormalTok{(prejpathmodel3, }\DataTypeTok{data =}\NormalTok{ data)}
      \KeywordTok{summary}\NormalTok{(prejpathfit4, }\DataTypeTok{fit.measures =}\NormalTok{ T)}
\end{Highlighting}
\end{Shaded}

\begin{verbatim}
## lavaan 0.6-3 ended normally after 297 iterations
## 
##   Optimization method                           NLMINB
##   Number of free parameters                         57
## 
##                                                   Used       Total
##   Number of observations                          1523        1860
## 
##   Estimator                                         ML
##   Model Fit Test Statistic                    1114.722
##   Degrees of freedom                               294
##   P-value (Chi-square)                           0.000
## 
## Model test baseline model:
## 
##   Minimum Function Test Statistic             6745.620
##   Degrees of freedom                               325
##   P-value                                        0.000
## 
## User model versus baseline model:
## 
##   Comparative Fit Index (CFI)                    0.872
##   Tucker-Lewis Index (TLI)                       0.859
## 
## Loglikelihood and Information Criteria:
## 
##   Loglikelihood user model (H0)              46244.743
##   Loglikelihood unrestricted model (H1)      46802.104
## 
##   Number of free parameters                         57
##   Akaike (AIC)                              -92375.486
##   Bayesian (BIC)                            -92071.765
##   Sample-size adjusted Bayesian (BIC)       -92252.839
## 
## Root Mean Square Error of Approximation:
## 
##   RMSEA                                          0.043
##   90 Percent Confidence Interval          0.040  0.045
##   P-value RMSEA <= 0.05                          1.000
## 
## Standardized Root Mean Square Residual:
## 
##   SRMR                                           0.043
## 
## Parameter Estimates:
## 
##   Information                                 Expected
##   Information saturated (h1) model          Structured
##   Standard Errors                             Standard
## 
## Latent Variables:
##                            Estimate  Std.Err  z-value  P(>|z|)
##   agency_scale =~                                             
##     agency1                   1.000                           
##     agency2                   0.084    0.087    0.970    0.332
##     agency3                   1.120    0.081   13.829    0.000
##     agency4                   1.065    0.074   14.404    0.000
##   self_efficacy_scale =~                                      
##     self_eff1                 1.000                           
##     self_eff2                 1.210    0.085   14.269    0.000
##     self_eff3                 1.115    0.078   14.253    0.000
##     self_eff4                 0.819    0.063   12.937    0.000
##     self_eff5                 1.023    0.068   15.160    0.000
##     self_eff6                 0.848    0.066   12.805    0.000
##     self_eff7                 0.937    0.069   13.567    0.000
##     self_eff8                 1.226    0.081   15.128    0.000
##   mental_wellness_scale =~                                    
##     sdq1                      1.000                           
##     sdq2                      1.043    0.064   16.260    0.000
##     sdq3                      1.270    0.073   17.494    0.000
##     sdq4                      1.064    0.068   15.678    0.000
##     sdq5                      1.242    0.072   17.275    0.000
##   risky_behaviour_scale =~                                    
##     FRNSMKR5                  1.000                           
##     FRNALCR5                  1.327    0.075   17.585    0.000
##     YOUALCR5                  1.215    0.080   15.234    0.000
##     BEATEN                    0.622    0.059   10.495    0.000
##     ARRSTDR5                  0.104    0.010   10.030    0.000
##     FRNGNGR5                  0.709    0.047   14.922    0.000
##     MEMGNGR5                  0.158    0.015   10.509    0.000
##     CRYWPNR5                  0.187    0.025    7.610    0.000
##     NUMPRTR5                  0.741    0.059   12.492    0.000
## 
## Regressions:
##                           Estimate  Std.Err  z-value  P(>|z|)
##   risky_behaviour_scale ~                                    
##     self_ffccy_scl           0.009    0.046    0.189    0.850
##     mntl_wllnss_sc          -0.179    0.022   -8.136    0.000
##   mental_wellness_scale ~                                    
##     self_ffccy_scl           1.195    0.232    5.143    0.000
##     agency_scale            -0.445    0.097   -4.597    0.000
##   self_efficacy_scale ~                                      
##     agency_scale             0.307    0.026   11.988    0.000
## 
## Variances:
##                    Estimate  Std.Err  z-value  P(>|z|)
##    .agency1           0.015    0.001   21.840    0.000
##    .agency2           0.053    0.002   27.584    0.000
##    .agency3           0.020    0.001   22.379    0.000
##    .agency4           0.015    0.001   20.927    0.000
##    .self_eff1         0.003    0.000   25.085    0.000
##    .self_eff2         0.004    0.000   24.553    0.000
##    .self_eff3         0.003    0.000   24.570    0.000
##    .self_eff4         0.003    0.000   25.585    0.000
##    .self_eff5         0.002    0.000   23.405    0.000
##    .self_eff6         0.003    0.000   25.661    0.000
##    .self_eff7         0.003    0.000   25.166    0.000
##    .self_eff8         0.003    0.000   23.457    0.000
##    .sdq1              0.012    0.001   22.860    0.000
##    .sdq2              0.013    0.001   22.880    0.000
##    .sdq3              0.012    0.001   20.353    0.000
##    .sdq4              0.016    0.001   23.607    0.000
##    .sdq5              0.013    0.001   20.967    0.000
##    .FRNSMKR5          0.004    0.000   22.015    0.000
##    .FRNALCR5          0.004    0.000   19.166    0.000
##    .YOUALCR5          0.008    0.000   23.771    0.000
##    .BEATEN            0.006    0.000   26.389    0.000
##    .ARRSTDR5          0.000    0.000   26.520    0.000
##    .FRNGNGR5          0.003    0.000   24.070    0.000
##    .MEMGNGR5          0.000    0.000   26.384    0.000
##    .CRYWPNR5          0.001    0.000   27.038    0.000
##    .NUMPRTR5          0.005    0.000   25.646    0.000
##     agency_scale      0.007    0.001    9.509    0.000
##    .self_ffccy_scl    0.000    0.000    6.749    0.000
##    .mntl_wllnss_sc    0.005    0.001   10.078    0.000
##    .risky_bhvr_scl    0.002    0.000   10.796    0.000
\end{verbatim}

The summary is still the same as that of the previous model, now let's
see the diagram

\begin{Shaded}
\begin{Highlighting}[]
\NormalTok{semPlot}\OperatorTok{::}\KeywordTok{semPaths}\NormalTok{(prejpathfit4, }\DataTypeTok{what =} \StringTok{"est"}\NormalTok{, }\DataTypeTok{layout =} \StringTok{"spring"}\NormalTok{)}
\end{Highlighting}
\end{Shaded}

\includegraphics{Test_2_files/figure-latex/unnamed-chunk-9-1.pdf} We can
now see that the risky behaviour scale is now icluded in the
relationship tree. The model shows that agency is a predictor of both
self-efficacy and mental wellnenss, and mental illness is in tern a
predictor of risky behaviour

\begin{verbatim}
                            Question 2
\end{verbatim}

2.1 Loading datasets

\begin{Shaded}
\begin{Highlighting}[]
\NormalTok{ethp  <-}\StringTok{ }\KeywordTok{load}\NormalTok{(}\StringTok{"YL_June2017_Ethiopiadata_2017-09-08"}\NormalTok{)}
\NormalTok{india <-}\StringTok{ }\KeywordTok{load}\NormalTok{(}\StringTok{"YL_June2017_Indiadata_2017-11-22"}\NormalTok{)}
\NormalTok{peru  <-}\StringTok{ }\KeywordTok{load}\NormalTok{(}\StringTok{"YL_June2017_perudata_2017-11-12"}\NormalTok{)}
\NormalTok{viet  <-}\StringTok{ }\KeywordTok{load}\NormalTok{(}\StringTok{"YL_June2017_Vietnamdata_2017-11-01"}\NormalTok{)}
\end{Highlighting}
\end{Shaded}

Now merging datasets

\begin{Shaded}
\begin{Highlighting}[]
\KeywordTok{remove}\NormalTok{(alldata)}
\end{Highlighting}
\end{Shaded}

\begin{verbatim}
## Warning in remove(alldata): object 'alldata' not found
\end{verbatim}

\begin{Shaded}
\begin{Highlighting}[]
\NormalTok{mylist <-}\StringTok{ }\KeywordTok{list}\NormalTok{( }\DataTypeTok{one=}\NormalTok{Ethiopia.dat, }\DataTypeTok{two=}\NormalTok{India.dat, }\DataTypeTok{three=}\NormalTok{peru.dat, }\DataTypeTok{four=}\NormalTok{ Vietnam.dat )}
\NormalTok{joined <-}\StringTok{ }\KeywordTok{join_all}\NormalTok{( mylist, }\DataTypeTok{type=}\StringTok{"full"}\NormalTok{ ) }
\end{Highlighting}
\end{Shaded}


\end{document}
